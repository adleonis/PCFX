
\documentclass[12pt]{article}
\usepackage[a4paper,margin=1in]{geometry}
\usepackage{hyperref}
\usepackage{graphicx}
\usepackage{titlesec}
\usepackage{enumitem}
\usepackage{setspace}
\usepackage{xcolor}

\titleformat{\section}{\large\bfseries}{\thesection}{1em}{}
\titleformat{\subsection}{\normalsize\bfseries}{\thesubsection}{1em}{}
\setlist{nosep}
\setstretch{1.15}

\title{\textbf{The Personal Cognitive Firewall eXchange Protocol (PCF-X)}\\
\vspace{0.3em}
\large A Universal Framework for Individual Cognitive Security}

\author{
\textbf{Whitepaper v1.0 — Draft for Public Consultation}\\[1em]
Stephane Gallet (Concept Lead)\\
\texttt{©2025 — CC BY-SA 4.0}
}

\date{}

\begin{document}
\maketitle
\begin{abstract}
Human civilization has entered an age where \textit{information itself has agency}.  
Large language models, synthetic media, and algorithmic recommendation engines generate persuasive content at a scale and speed no human cognition can match.  
In this environment, the self becomes the final unsecured endpoint.

The \textbf{Personal Cognitive Firewall eXchange Protocol (PCF-X)} establishes an open, decentralized standard for capturing, describing, and analyzing the informational inputs that shape individual thought.  
It defines a modular protocol—spanning ingestion, analysis, and presentation—designed for privacy-preserving interoperability between devices, platforms, and independent developers.

Its goal is to make cognitive influence visible, measurable, and owned by the individual.  
PCF-X is not a product: it is an epistemic infrastructure for the digital age, a public good analogous to HTTPS or SMTP—securing not communication, but \textit{comprehension itself}.
\end{abstract}

\section{Background: From Cybersecurity to Cognitive Security}

\subsection{The Erosion of Cognitive Boundaries}
Human attention has become the most valuable and contested resource of the twenty-first century.  
Recommendation algorithms, targeted advertising, and now generative models continuously adjust content to optimize persuasion.  
These mechanisms operate invisibly—adapting faster than any individual’s capacity for reflection.  
The result is a new class of threat: \textbf{cognitive intrusion}.

\subsection{The Limits of Existing Frameworks}
Privacy regulations (GDPR, CCPA) protect data ownership; cybersecurity frameworks protect device integrity.  
But neither addresses cognitive integrity.  
There is no protocol ensuring that individuals can \textbf{see and audit the informational forces shaping their worldview}.

\subsection{The Need for an Open Protocol}
If each company or government implements its own opaque “attention defense,” cognitive security will fragment into competing silos.  
What humanity needs is a \textbf{shared, auditable protocol} ensuring cognitive autonomy remains a public standard, not a corporate feature.

\section{Foundational Philosophy}

\subsection{Post-Truth and the Architecture of Perception}
As Hannah Arendt warned, “The ideal subject of totalitarian rule is not the convinced Nazi or Communist, but people for whom the distinction between fact and fiction no longer exists.”  
In the digital era, this condition is automated.

\subsection{From Subjectivity to Transparency}
Following Kuhn, Latour, and Bayesian epistemology, PCF-X accepts that all knowledge is perspectival.  
Rather than enforcing “objective truth,” it exposes the \textbf{topology of influence}—mapping how narratives and emotions interact within each user’s informational field.

\subsection{Digital Autonomy as a Human Right}
If privacy was the right to silence, cognitive transparency is the right to awareness.  
PCF-X gives individuals instruments to observe, without intermediaries, the informational patterns acting upon them.

\section{The Vision: A Universal Cognitive Defense Layer}
PCF-X proposes a three-layer framework:

\begin{center}
\begin{tabular}{|l|l|l|}
\hline
\textbf{Layer} & \textbf{Function} & \textbf{Analogy}\\
\hline
Information Ingestion & Capture human-facing informational input & Network Interface\\
Data Analysis & Transform input into semantic, emotional, relational atoms & Transport/Interpretation\\
Presentation & Render analytics and controls for the individual & Application\\
\hline
\end{tabular}
\end{center}

Each layer is modular and independently implementable.  
The protocol defines data contracts and consent primitives enabling secure interoperability.

\section{Design Principles}
\begin{enumerate}
\item \textbf{Local-first sovereignty}: computation on user device.  
\item \textbf{Modularity and extensibility}: independent development encouraged.  
\item \textbf{Capability-based security}: least privilege per component.  
\item \textbf{Human-readable transparency}: every result links to evidence.  
\item \textbf{Interoperability over ownership}: protocol $>$ platform.  
\item \textbf{Ethical universality}: respect for local privacy norms.  
\end{enumerate}

\section{Protocol Overview}

\subsection{Core Data Model}
Artifacts include:
\begin{itemize}
\item \textbf{ExposureEvent} — informational exposure (app use, ad view, audio segment)
\item \textbf{KnowledgeAtom} — extracted claim with entities, tones, sources
\item \textbf{Relation} — linkage (similarity, contradiction, coordination)
\item \textbf{Metric} — aggregate analytic result
\item \textbf{ConsentManifest} — human-signed capability grants
\end{itemize}

All artifacts are signed, timestamped, and stored in a \textbf{Personal Data Vault (PDV)}.

\subsection{System Flow}
\begin{verbatim}
[ Ingestion Adapters ] → [ Event Bus ] → [ Analysis Nodes ] → [ Presentation Clients ]
\end{verbatim}

\subsection{Governance of Data Flow}
Adapters emit only \textit{ExposureEvents}; analysis nodes produce \textit{KnowledgeAtoms} and \textit{Relations}; presentation clients are read-only.  
The PDV enforces retention and cryptographic integrity.

\section{Privacy and Consent Framework}
PCF-X uses a capability-grant model inspired by object-capability security and Kantara consent receipts.  
Every component declares capabilities (e.g., \texttt{microphone.capture}).  
Users issue signed, time-limited \textit{ConsentManifests}.  
The PDV enforces consent at runtime.

\section{Technical Reference Architecture}

\subsection{Personal Data Vault}
Append-only log and column store with vector index; user-owned keys.  
APIs:  
\texttt{/events}, \texttt{/atoms}, \texttt{/relations}, \texttt{/metrics}.  

\subsection{Event Bus}
Local gRPC/WebSocket pub-sub; localhost only by default.  
Optional encrypted federation for aggregated metrics.

\subsection{Runtime and Sandboxing}
Preferred runtime: WebAssembly with restricted WASI; outbound networking requires explicit capability.

\section{Example Workflow}
\begin{enumerate}
\item Android adapter captures app focus, emits \texttt{ExposureEvent}.  
\item Atomizer node converts text/audio into \texttt{KnowledgeAtoms}.  
\item Coordination node detects repeated phrasing, emits \texttt{Relation}.  
\item Metrics node computes exposure velocity and influence index.  
\item Dashboard renders the daily influence report.
\end{enumerate}

\section{Future Directions}
\subsection{Federated Cognitive Weather System}
Anonymized metrics from multiple PDVs could form real-time maps of global influence flows.  

\subsection{Integration with LLMs}
Local agents can explain: “This article repeats phrasing seen 23 times this week.”  

\subsection{Civic and Research Use}
Academics and journalists can study media ecosystems with consent.  

\subsection{Hardware Integrations}
Smart glasses, vehicles, TVs—any surface emitting \texttt{ExposureEvents} can join the protocol.

\section{Governance and Participation}
\subsection{PCF-X Consortium}
A nonprofit standards body similar to W3C/IETF.  
Working groups: Ingestion, Analysis, Privacy, UX.

\subsection{Open-Source Reference Stack}
\texttt{pcfx-spec}, \texttt{pcfx-core}, \texttt{pcfx-sdk} repositories.

\subsection{Stakeholders}
Academia, industry, civil society, governments, and open-source developers.

\section{Why It Matters}
\subsection{The Next Layer of the Internet}
HTTP transferred information, HTTPS secured it, PCF-X will make it intelligible.  

\subsection{Restoring the Social Contract of Knowledge}
When every feed is personalized, common reality fragments.  
PCF-X restores minimal transparency for pluralism without chaos.  

\subsection{Human Augmentation, not Control}
It equips individuals with mirrors, not filters.

\section{Philosophical References}
Kuhn (1962), Latour (1987), Wiener (1948), Simon (1957), McLuhan (1964), Arendt (1951), Ostrom (1990).

\section{Technical References}
Miller et al. (Object Capability Security), Cavoukian (Privacy by Design, 2011), Berners-Lee (Solid Project),  
IETF JSON Schema, W3C DID, McMahan \& Ramage (Federated Learning, 2017).

\section{Call to Action}
Without transparent tools for self-observation, humans risk becoming programmable substrates for algorithmic economies.  
PCF-X invites engineers, philosophers, and citizens to secure the most valuable system we possess: the human mind in a networked world.

\begin{quote}
\textit{Let there be a standard for awareness.  
Let cognition itself have a protocol.}
\end{quote}

\bigskip
\noindent\textbf{Contact and Stewardship (proposed):}\\
\textit{PCF-X Working Group — Open founding members sought}\\
\texttt{https://pcfx.org (placeholder)}

\end{document}

